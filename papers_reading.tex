\documentclass{article}

\usepackage[english]{babel}
\usepackage[utf8]{inputenc}
\usepackage{amsmath}
\usepackage{amsfonts}
\usepackage{dsfont}
\usepackage{graphicx}
\usepackage{tikz}
\usepackage[colorinlistoftodos]{todonotes}
\usepackage{enumerate}




\title{Papers I'm reading and ideas about them}


\author{Bruno Seefeld}

\newtheorem{theorem}{Theorem}[section]
\newtheorem{corollary}{Corollary}[theorem]
\newtheorem{lemma}[theorem]{Lemma}
\newtheorem{definition}{Definition}[section]






\begin{document}


\maketitle


\begin{abstract}
List of papers I'm reading and important ideas about them.
\end{abstract}



\section{The Burnside problem for Diff$_\omega^\infty(\mathbb{S}^2)$}


\subsection{Introduction to the Burnside problem}

Let $G$ be a finitely generated group such that every element has finit
order, is $G$ necessarly finite? The answer in general is no.

The paper focuses on three different cases of the following question:
if the order of the above group is bounded, is the group finite.

The cases are:

\subsection{$G<\text{Diff}_\omega^\infty(\mathbb{S}^2)$}

$G$ is a subgroup of area preserving diffeomorphisms. In this
case the answer to the last question is ''YeS"". 
In order to prove it one considers $S=\{s_1,\ldots,s_n\}$ a symmetric
set of generators of the group, the shift space $\Sigma=S^\mathbb{Z}$:
$F:\Sigma\times M \to \Sigma\times M$ given by $F(w,x)=(\sigma w,w_0 x)$.
Since every element of $G$ has finite order, the cocycle cannot admit
hyperbolic fixed points. 

A group without hyperbolic fixed points is called elliptic.
For $G$ elliptic, one shows that the growth of derivatives of $G$ 
is subexponential. In order to do this on asssumes the contrary and
constructs a hyperbolic invariant ergodic measure and applies the 
following theorem

\begin{theorem}[Katok]
    IF $f$ is a $C^{1+\alpha}$ diffeomorphism on a compact smooth manifold and 
    $\mu$ is a hyperbolic ergodic invariant measure, then $f$ admits
    hyperbolic periodic points and we can take them close to supp$(\mu)$.

\end{theorem}










\end{document}
