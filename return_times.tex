\documentclass{article}

\usepackage[english]{babel}
\usepackage[utf8]{inputenc}
\usepackage{amsmath}
\usepackage{amsfonts}
\usepackage{dsfont}
\usepackage{graphicx}
\usepackage{tikz}
\usepackage[colorinlistoftodos]{todonotes}
\usepackage{enumitem}




\title{Return times in infinite ergodic theory}


\author{Bruno Seefeld}

\newtheorem{theorem}{Theorem}[section]
\newtheorem{corollary}{Corollary}[theorem]
\newtheorem{lemma}[theorem]{Lemma}
\newtheorem{definition}{Definition}[section]
\newtheorem{proposition}{Proposition}[section]



\begin{document}
\maketitle


\section{Examples}

\subsection{Bernoulli shift}
Let $X=\{-1,1\}^\mathbb{Z}$ and $\phi:X\to \mathbb{R}$ be given by 
$\phi(x_n)_n=x_0$. Take $U=\{x\in X| x_0=1\}$ which we denote by $[0:1]$.
We have, for $n>0$, $\sigma^{-n}(U)\cap U=[0:1, n:1]$. When $n$ is even,
$\sigma^{-n}(U)\cap U\cap \{|S_\phi(n,)|<\epsilon\}=\emptyset$; when
$n\geq 3$ is odd, 
$\sigma^{-n}(U)\cap U\cap |\{S_\phi(n,)|<\epsilon\}=[0:1, W, n:1]$
with $W$ being a block of length $n-1$ summing to $-2$. 

The measure of each block is $\frac{1}{2}^{n+1}$ and there are 
${n-1\choose \frac{n+1}{2}}$ such blocks, therefore the intersection
has measure  $B_n=\frac{1}{2}^{n+1} {n-1\choose \frac{n+1}{2}}$, so
that the set $R(U)$ is syndetic. Since $B_n\to 0$, the set $R_\lambda(U)$
is not syndetic though.


\subsection{Some irrational rotations}



\section{The general case}

Let $(X,\mathcal{B},\mu)$ be an infinite measure $\sigma$-
finite measure space and $T$ a measure preserving conservative ergodic
transformation. We have: 

\begin{theorem}[Hopf ratio ergodic theorem]
For $f,g\in L_1(X,\mu)$, $g\geq 0$,
\begin{align*}
    \frac{S_n(f)}{S_n(g)}\to \frac{\int f d\mu}{\int g f\mu} \quad \text{a.e on} \quad X
\end{align*}
\end{theorem}

and by taking a sequence of sets $A_n\uparrow X$ and $g_N=\mathds{1}_{A_N}$
we get 

\begin{corollary}[Birkhoff ergodic theorem]
For $f\in L_1(X,\mu)$
\begin{align*}
        \frac{1}{n}S_n(f)\to 0 \quad \text{a.e on} \quad X.
\end{align*}
\end{corollary}

So by taking a set $A\in \mathcal{B}$ with $0<\mu(A)<\infty$ we 
can apply the Dominated Convergence Theorem and get that 
$\frac{1}{n}\sum_{k=0}^{n-1}\mu(A\cap T^{-k} A)\to 0$.
Therefore the set of $\lambda$-large return times of $A$ cannot be syndetic,
since if there was a gap $\tau$ we would have for all $n$ large
$\frac{1}{n}\sum_{k=0}^{n-1}\mu(A\cap T^{-k} A)\geq \frac{1}{\tau}\lambda\mu(A)^2>0$,
a contradiction.

\subsection{Skew-products}

For a probability space $(X,\mathcal{B},\mu)$, an ergodic 
measure preserving transformation $T$, and an integrable
function $\phi\in L_1(X,\mu)$ we can define the \textbf{skew-product}
$F_\phi: X\times \mathbb{R}\to X\times\mathbb{R}$ by 

\begin{align*}
F_\phi(x,t)=(T(x),t+\phi(x))
\end{align*}

with the invariant measure being the product of $\mu$ and Lebesgue.
\paragraph{}

The second coordinate of the iterates of $F_\phi$ is given by 
$S_n(x)$ for $n>0$ and $\sum_{j=-1}^{-n}\phi(T^{j}x)$ for $n<0$.
There is nothing special with the group $\mathbb{R}$, we could 
change it by any locally compact group and it's Haar measure. 

\paragraph{}

With this last information in mind consider $\phi:X\to \mathbb{Z}$
with zero integral. 

\paragraph{}

=======next writing
-Atkinson for $\mathbb{Z}$
-atkinson showed that a certain set is infinite
-the results show that it can't be syndetic for a Khitnchine 
type bound

-can it be true if we remove the bound?.




























\end{document}