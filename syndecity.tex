\documentclass{article}

\usepackage[english]{babel}
\usepackage[utf8]{inputenc}
\usepackage{amsmath}
\usepackage{amsfonts}
\usepackage{dsfont}
\usepackage{graphicx}
\usepackage{tikz}
\usepackage[colorinlistoftodos]{todonotes}
\usepackage{enumerate}




\title{Syndecity of the set ot return times for skew-products over
ergodic systems}


\author{Bruno Seefeld}

\newtheorem{theorem}{Theorem}[section]
\newtheorem{corollary}{Corollary}[theorem]
\newtheorem{lemma}[theorem]{Lemma}
\newtheorem{definition}{Definition}[section]






\begin{document}


\maketitle


\begin{abstract}

\end{abstract}


\section{Introduction}

Let $(X,\mathcal{X},\mu)$ be a probability space and $T$ a mpt.
For an integrable function $f\in L_1(X,\mu)$ we define for each $n\in
\mathbb{N}$ the cocycle $a_f(n,x)=\sum_{i=0}^{n-1}f(T^j x)$.
This cocycle is important when studying the skew-product over $X$,
i.e, the measurable transformation $S_f:X\times\mathbb{R}\to X\times\mathbb{R}$
given by:

\begin{equation}
S_f(x,t)=(T(x),t+a(1,x))
\end{equation}


the iterates of a point $(x,t)$ will be $S_f^n(x,t)=(T^n(x),a_f(n,x))$.


We say that the cocycle $a_f$ is recurrent when for all measurable
sets of positive $\mu$ measure $A$ we have: for $\epsilon>0$ we can
find $n\in\mathbb{N}$ with $\{T^{-n}(A)\cap A\cap \{|a_f(n,)|<\epsilon\}\}$.
There is to say, not only we have recurrence for the dynamics $T$ but
the points get back with small sum. 

If the dynamical system $(X,T)$ is ergodic and $f$ has zero average,
then we can expect that the orbits visit the positive and negative 
regions in a balanced way so that we can comeback with small sum (this is
more intuitive when we imagine $f$ bounded). In fact this is true by
the following theorem of Atkinson:


\begin{theorem}[Atkinson]
    For an atomless probability space $(X,\mathcal{X},\mu)$, an ergodic translation $T$,
    and $f$ integrable, $a_f$ is recurrent iff $\int f d\mu=0$. 

\end{theorem}


Let $\nu$ be the product measure $\mu\times \text{Leb}$. We're interested in the recurrence times for a set $A\times B\subset 
X\times \mathbb{R}$,i.e, the set $R_\lambda (A\times B)=\{n\in\mathbb{N}|
\nu(S_f^{-n}(A\times B)\cap(A\times B))>\lambda \nu (A\times B)^2\}$.  This
are the times that the sets comes back to itself bu with positive  measure.
For the non skew-product case, with only a a measure preserving transformation
on a finite measure space Kintchine theorem says that $R_\lambda(A)$
is syndetic, the proof is an aplication of von-Neumann ergodic theorem.














\end{document}
