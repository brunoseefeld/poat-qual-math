\documentclass{article}

\usepackage[english]{babel}
\usepackage[utf8]{inputenc}
\usepackage{amsmath}
\usepackage{amsfonts}
\usepackage{dsfont}
\usepackage{graphicx}
\usepackage{tikz}
\usepackage[colorinlistoftodos]{todonotes}
\usepackage{enumerate}




\title{Syndecity of the set ot return times for skew-products over
ergodic systems}


\author{Bruno Seefeld}

\newtheorem{theorem}{Theorem}[section]
\newtheorem{corollary}{Corollary}[theorem]
\newtheorem{lemma}[theorem]{Lemma}
\newtheorem{definition}{Definition}[section]
\newtheorem{proposition}{Proposition}[section]







\begin{document}


\maketitle


\begin{abstract}

\end{abstract}


\section{Introduction}

Let $(X,\mathcal{X},\mu)$ be a probability space and $T$ a mpt.
For an integrable function $f\in L_1(X,\mu)$ we define for each $n\in
\mathbb{N}$ the cocycle $a_f(n,x)=\sum_{i=0}^{n-1}f(T^j x)$.
This cocycle is important when studying the skew-product over $X$,
i.e, the measurable transformation $S_f:X\times\mathbb{R}\to X\times\mathbb{R}$
given by:

\begin{equation}
S_f(x,t)=(T(x),t+f(x))
\end{equation}


the iterates of a point $(x,t)$ will be $S_f^n(x,t)=(T^n(x),t+a_f(n,x))$.
\\

We say that the cocycle $a_f$ is recurrent when for all measurable
sets of positive $\mu$ measure $A$ we have: for $\epsilon>0$ we can
find $n\in\mathbb{N}$ with $\{T^{-n}(A)\cap A\cap \{|a_f(n,)|<\epsilon\}\}$.
There is to say, not only we have recurrence for the dynamics $T$ but
the points get back with small sum. 

If the dynamical system $(X,T)$ is ergodic and $f$ has zero average,
then we can expect that the orbits visit the positive and negative 
regions in a balanced way so that we can comeback with small sum (this is
more intuitive when we imagine $f$ bounded). In fact this is true by
the following theorem of Atkinson:


\begin{theorem}[Atkinson] \label{theorem:Atkinson}
    For an atomless probability space $(X,\mathcal{X},\mu)$, an ergodic translation $T$,
    and $f$ integrable, $a_f$ is recurrent iff $\int f d\mu=0$.
\end{theorem}


Let $\nu$ be the product measure $\mu\times \text{Leb}$. We're interested in the recurrence times for a set $A\times B\subset 
X\times \mathbb{R}$,i.e, the set $R_\epsilon (A\times B)=\{n\in\mathbb{N}|
\nu(S_f^{-n}(A\times B)\cap(A\times B))>\nu (A\times B)^2-\epsilon\}$.  This
are the times that the set comes back to itself but with large  measure.
For the non skew-product case, with only a measure preserving transformation
on a finite measure space Kintchine theorem says that $R_\epsilon(A)$
is syndetic, the proof is an aplication of von-Neumann ergodic theorem.
(ref petersen).
\\

One may ask if in the infinite invariant measure we have an
analogous Khintchine theorem. In the skew-product
setting:
\paragraph{}
\textbf{Problem:} Given $(X\times \mathbb{R}, \nu,S_f)$ with $a_f$
recurrent, $\epsilon>0$ and $A\subset X\times \mathbb{R}$ of positive
measure, is the set $R_\epsilon(A)=\{n\in \mathbb{N}|\nu(S_f^{-n}
A\cap A)>\nu(A)^2-\epsilon\}$ syndetic?

In the next section we show that an example by Aaronson (discrepancy 
skew product over badly approximable irrational numbers) gives a negative
answer to that question. 


\section{Discrepancy skew products over irrational rotations}


Consider $\alpha\in \mathbb{R}\setminus \mathbb{Q}$
a badly approximable irrational, there is:
\begin{align*}
    \inf\{q^2|\alpha-\frac{p}{q}| :\quad q\in\mathbb{N}^+, p\in \mathbb{Z}\}>0.
\end{align*}
we'll consider the example from  (ref Aaronson), called \textit{discrepancy
skew product over} $\alpha$:
\begin{align*}
T:\mathbb{T}\to \mathbb{T}&\\
 x\mapsto & x+\alpha \\
f:\mathbb{T}\to \mathbb{Z}&\\
x\mapsto & 2\mathds{1}_{[0,1/2)}x-\mathds{1}_{[1/2,0)}x.
\end{align*}

Clearly this satisfyes the hypothesis of Atkinson theorem \ref{theorem:Atkinson}
considering $\nu$ the product of Lebesgue and the counting measure on $\mathbb{Z}$.

Let $A=\mathbb{T}\times {0}$, $\epsilon>0$. Notice that $(x,0)
\in S_f^{-n}(A)\cap A$ iff $x\in \{y\in \mathbb{T}: a_f(n,y)=0\}$, 
therefore $\nu(S_f^{-n}A\cap A)=\mu(\{a_f(n,;)=0\})$.


The main result of the paper is that the dynamical system 
$(\mathbb{T}\times \mathbb{Z},S_f)$ is \textit{boundedly rationally ergodic}.
Denote $\Psi_n: \mathbb{T}\to \mathbb{N}$ the function:
\begin{align}
    \psi(x)=&\sum_{k=1}^{n-1} \mathds{1}_{\mathbb{T}\times{0}} S_f(x,0)=\\
    & \#\{1\leq k\leq n-1: a_f(k,x)=0\}
\end{align}

a corollary of the  main result of (ref paper) is:

\begin{corollary}\label{corollary:Aaronson}
There exists $M>1$ with $\int_\mathbb{T} \Psi_n=M^{\pm}\frac{n}{\log^{\frac{1}{2}}n}$. 
\end{corollary}

with $x=M^\pm y$ meaning $\frac{1}{M}\leq \frac{x}{y}\leq M$.
\paragraph{}
When $\alpha$ is quadratic we have, from the paper (ref old paper
by Aaronson), that:

\begin{equation}\label{equation:Aaronson}
    \sum_{k=1}^{n}\nu (S_f^{-k}A\cap A)=\int_\mathbb{T}\Psi_n
\end{equation}


\begin{proposition}
For $\alpha$ a quadratic irrational and $(\mathbb{T}\times\mathbb{Z},
S_f)$ the discrepancy skew product over $\alpha$, for $\epsilon>0$ small
enough, $A=\mathbb{T}\times \{0\}$ the set $R_{\epsilon}(A)$ is not
syndetic.
\end{proposition}

\textbf{Proof:} Assume that the $R_\epsilon(A)$ is
syndetic. Then there exists $\tau>1$ such that for all $j\in\mathbb{N}$, $R_\epsilon(A)\cap
\{j,j+1,\ldots,j+\tau\}\neq \emptyset$ . This means that for $N$
large, $\# R_\epsilon\cap \{1,\ldots,N\}\geq \frac{N}{\tau}$.

Notice that 
\begin{align}
    \sum_{k=1}^{n}\nu (S_f^{-k}A\cap A)\geq& (1-\epsilon)\#R_\epsilon\cap \{1,\ldots,n\}\geq\\
    (1-\epsilon)\frac{n}{\tau}>& M \frac{n}{\log^{\frac{1}{2}}n}\quad \text{for}\quad n\quad \text{large enough}      
\end{align}

combining the last inequality with formula
 \ref{equation:Aaronson} and the corollary \ref{corollary:Aaronson}
 we get a contradiction.
 
 

 \section{Further questions}






































\end{document}
