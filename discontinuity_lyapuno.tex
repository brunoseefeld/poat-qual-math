\documentclass{article}

\usepackage[english]{babel}
\usepackage[utf8]{inputenc}
\usepackage{amsmath}
\usepackage{amsfonts}
\usepackage{dsfont}
\usepackage{graphicx}
\usepackage{tikz}
\usepackage[colorinlistoftodos]{todonotes}
\usepackage{enumerate}




\title{Some results about cocycles}
\author{Bruno Seefeld}




\newtheorem{theorem}{Theorem}[section]
\newtheorem{corollary}{Corollary}[theorem]
\newtheorem{lemma}[theorem]{Lemma}
\newtheorem{definition}{Definition}[section]






\begin{document}


\maketitle


\begin{abstract}
Some interesting exercises/resutls about cocycles
\end{abstract}



\section{Non-continuity of Lyapunov exponents}

Given two matrices $A_1,A_2\in \text{GL}_2(\mathbb{R})$ and 
numbers $p_1,p_2$ with $p_1+p_2=1,p_1,p_2>0$, we can consider
the random matrix $A^(n)=A_{i_n}\ldots A_{i_1}$ with $i_k\in\{1,2\}$
with a Bernoulli $B(p_1)$ distribution. We can consider the assymptotic
exponential rate of growth of the  norm of $A^{(n)}$ given by 
\begin{equation}
\lambda^+(A_1,A_2,p_1,p_2)=\lim_{n\to \infty}\frac{1}{n}\log \|A^{(n)}\|.
\end{equation}

The following theorem \footnote{This is true for dimension d and d matrices} says that the quantity $\lambda^+$ is continuous
on $\text{GL}_2(\mathbb{R})$ and $(0,1)^2$:

\begin{theorem}[Bocker-Viana]
    The function $\lambda^+: \text{GL}_d(\mathbb{R})\times \text{GL}_d(\mathbb{R}) \times (0,1)^2\to \mathbb{R} $
    is continuous. 
\end{theorem}


What happens at $p_1=1$? We'll show an example where we don't have continuity
of $\lambda^+$ at this point. 

Let's take

\begin{align*}
    A_1=&
    \begin{pmatrix}
        2 & 0 \\
        0 & 1/2
    \end{pmatrix}\\
    A_2=&
    \begin{pmatrix}
        0 & -1 \\
        1 & 0
    \end{pmatrix}\\
\end{align*}

For $p_1=1$ we have $A^{(n)}=A_1^n$ almost surely, therefore 
$\lambda^+(A_1,A_2,1,0)=\log 2$. In order to simplify notation denote $p_2=p$ and $X_n=\frac{1}{n}\log \|A^{(n)}\|$, we'll show that if $p>0$
then $\lambda^+(A_1,A_2,(1-p),p)=0$, and then there is no continuity at 
$(A_1,A_2,1,0)$. It's enough to show that the sequence of random variables $X_n$
converge in probability to 0, therefore it cannot converge almost surely to something
else.

By observing that  $A_1 A_2 A_1= A_2$ and $A_2^2=-Id$ we can write the norm of the products 
$\|A^{(n)}\|$ as  $\|A_1^{n_1}A_2^{n_2}\|$, with the number  $n_1$ being 
given by a random walk
\[
S_{n+1}=\begin{cases}
    S_n+ 1 & \text{if}\quad X_{n+1}=A_1 \\
    -S_n & \text{}if\quad X\_{n+1}=A_2
\end{cases}
\]













for example:

\begin{align}
    A^{10}=& A_1^4 A_2 A_1 A_2 A_1^3=  \label{product}\\
    A_1^4 A_2 (A_1 A_2 A_1) A_1^2=& A_1^4 A_2 A_2 A_1^2=\\
    A_1^4 (A_2^2) A_1^2=& A_1^6 A_2^2  \\
    \{S_n\}_{n=1}^{10}=&\{1,2,3,-3,-2,2,3,4,5,6\}
\end{align}





We have then $\frac{1}{n}\log\|A^{(n)}\|\leq \frac{n_1}{n}\log 2$.
The convergence in probability to 0 will be estabilished then by showing
$\mathbb{E}[|S_n|]=O(\sqrt{n})$.

Let $\xi_i $ be 1 if $X_i=A_2$ and 0 otherwise. For every product 
$A^{(n)}$ we can assositate a sequence of runs $\xi_1,\ldots, \xi_n$. 
For each of this runs we'll consider the random variables $\nu_k$ that 
count the number of zeros between 1's. More formally denote $\tau_0=0$ 
$\tau_k=\inf\{k>\tau_{k-1}| \xi_k=1\}$ the time that a one appears after
the last apperance of a 1 (in position $\tau_{k-1}$). We have then 
$\nu_k=\tau_k-\tau_{k-1}-1$.

For example: the run associated with the product  \ref{product} is
$\{0,0,0,1,0,1,0,0,0,0\}$ so that $\tau_1=3,\tau_2=5$ and $\nu_1=0,\nu_2=1$.

Notice that $\mathbb{P}[\nu_k\geq l]=(1-p)^l$, so that the independent
random variables $\nu_k$ are i.i.d geometrically distributed. We 
consider the process $T_m=S_{\tau_m -1}$. Depending on the parity of $m$
we have:

\begin{equation}
    T_m=\nu_1-\nu_2+\ldots \pm\nu_m.
\end{equation}

Therefore

\begin{align*}
    \mathbb{E}[T_m]=
    \begin{cases}
        0, & \text{if}\quad $m$ \quad \text{is even} \\
        \frac{1-p}{p} & \text{if}\quad $m$ \quad \text{is odd}
    \end{cases}
\end{align*}

and the variance is given by $\frac{1-p}{p^2}m$.

Applying the Chernoff bound we get, for any $\delta>0:$

\begin{equation*}
    \mathbb{P}(|T_m|>m^{1/2+\delta})\leq e^{-\frac{p^2}{2^{2\epsilon}(1-p)}
    m^{2\delta}}
\end{equation*}

for some $\epsilon$ depending only on $\delta$.

(why does this implies the appropriate order of growth?).























\end{document}
