\documentclass{article}

\usepackage[english]{babel}
\usepackage[utf8]{inputenc}
\usepackage{amsmath}
\usepackage{amsfonts}
\usepackage{dsfont}
\usepackage{graphicx}
\usepackage{tikz}
\usepackage[colorinlistoftodos]{todonotes}
\usepackage{enumerate}




\title{Notes on rotation theory}


\author{Bruno Seefeld}

\newtheorem{theorem}{Theorem}[section]
\newtheorem{corollary}{Corollary}[theorem]
\newtheorem{lemma}[theorem]{Lemma}
\newtheorem{definition}{Definition}[section]






\begin{document}


\maketitle


\begin{abstract}
Some important results in rotation theory in low dimensions.
\end{abstract}



\section{Plane homeomorphisms}

Plane homeomorphisms will be important to the study of anulli homeomorphisms,
a fundamental result used in other, more complicated proofs is this result by
Brower:

\begin{theorem}
    For $f:\mathbb{R}^2\to \mathbb{R}^2$ an orientation presenving
    homeomorphism, if $f$ has no periodic point then it has no fixed point.

\end{theorem}


Fundamental to the proof of this theorem is a result stating that if 
$f$ has a periodic disk chain, then f has a fixed point.


\begin{definition}[Periodic disk chain]
A periodic disk chain for $f$ is a family $U_1,\ldots, U_n$ os disks with 
\begin{enumerate}
    \item $f(U_i)\cap U_i=\emptyset$
    \item $U_i\cap U_j=\emptyset$ for $i\neq j$
    \item For all i, there exists $m_{i-1}$ such that $f^{m_{i-1}}U_{i-1}\cap 
    U_i \neq \emptyset$
\end{enumerate}
\end{definition}



One example of homeomorphism without fixed points is given by translations,
we have a strong result due to Brower that says that in a sense every homeomorphism
without fixed points acts like a translation.

For this we need:

\begin{definition}[Translation domain]
For a properly embedded line $L$, we say the open connected region whose
boundary is $L\cup f(L)$ is a translation domain for $f$ if $L$ separates
$f(L),f^{-1}(L)$.

\end{definition}


\begin{theorem}[Brower translation theorem]
    If $f$ is a plane homeomorphisms, then every point is contained in a 
    translation domain.
\end{theorem}


Let $D$ be a translation domain with boundaries $L,f(L)$, consider $U=\cup_{n\in
\mathbb{Z}}f^n(D)$, this set is connected (connect  a point to it's boundary, keep doing that till
you reach the other point), open (boundary points intersect only another translation domain 
because the line are properly embedded) and invariant. 
One can construct a homeomorphism $h_0: D\to \mathbb{R}\times [0,1]$ that takes
$L$ to $\mathbb{R}\times \{0\}$ and $f(L)$ to $\mathbb{R}\times\{1\}$ in such 
a way that $h(x)+(0,1)=h(f(x))$ for $x\in L$. By changing $[0,1]$ for $[n-1,n]$
we construct a homeomorphisms $h_n$ analogously, define $h:U\to \mathbb{R}^2$ 
by gluing all the $h_n's$. 




\section{Applicastions of Brower translation theorem}

\subsection{
    Orientation preserving omemorphims of the sphere $\mathbb{S}^2$ with one fixed point}

If $f:\mathbb{S}^2\to \mathbb{S}^2$ contains only one fixed point,
then by the stereographical projection with a have on the plane a 
homeomorphisms without fixed points, therefore without periodic points.
Going back to the spehre we have that there are no periodic points.
Also, by Brower translation theorem we have that everuy point is in a 
translation domain, hence non-recurrent. Since on the orbit of 
the translation domain the homeomorphisms acts like a translation,
we have that for any $x\in \mathbb{R}^2$, $\lim_{i}f^i(x)=\infty=\lim_{i}f^{-i}(x)$,
this implies Cr$(f)=\mathbb{S}^2$.

\subsection{Orientation preserving homemorphims of $\mathbb{S}^2$ wiht two fixed points}
By removing the two fixed points we can think the homeomorphism acts
on the open anullus $\mathbb{A}$. 













\end{document}
