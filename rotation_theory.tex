\documentclass{article}

\usepackage[english]{babel}
\usepackage[utf8]{inputenc}
\usepackage{amsmath}
\usepackage{amsfonts}
\usepackage{dsfont}
\usepackage{graphicx}
\usepackage{tikz}
\usepackage[colorinlistoftodos]{todonotes}
\usepackage{enumerate}




\title{Notes on rotation theory}




\newtheorem{theorem}{Theorem}[section]
\newtheorem{corollary}{Corollary}[theorem]
\newtheorem{lemma}[theorem]{Lemma}
\newtheorem{definition}{Definition}[section]






\begin{document}


\maketitle


\begin{abstract}
Some important results in rotation theory in low dimensions.
\end{abstract}



\section{Plane homeomorphisms}

Plane homeomorphisms will be important to the study of anulli homeomorphisms,
a fundamental result used in other, more complicated proofs is this result by
Brower:

\begin{theorem}
    For $f:\mathbb{R}^2\to \mathbb{R}^2$ an orientation presenving
    homeomorphism, if $f$ has no periodic point then it has no fixed point.

\end{theorem}


Fundamental to the proof of this theorem is a result stating that if 
$f$ has a periodic disk chain, then f has a fixed point.


\begin{definition}[Periodic disk chain]
A periodic disk chain for $f$ is a family $U_1,\ldots, U_n$ os disks with 
\begin{enumerate}
    \item $f(U_i)\cap U_i=\emptyset$
    \item $U_i\cap U_j=\emptyset$ for $i\neq j$
    \item For all i, there exists $m_{i-1}$ such that $f^{m_{i-1}}U_{i-1}\cap 
    U_i \neq \emptyset$
\end{enumerate}
\end{definition}



One example of homeomorphism without fixed points is given by translations,
we have a strong result due to Brower that says that in a sense every homeomorphism
without fixed points acts like a translation.

For this we need:

\begin{definition}[Translation domain]
For a properly embedded line $L$, we say the open connected region whose
boundary is $L\cup f(L)$ is a translation domain for $f$ if $L$ separates
$f(L),f^{-1}(L)$.

\end{definition}


\begin{theorem}[Brower translation theorem]
    If $f$ is a plane homeomorphisms, then every point is contained in a 
    translation domain.
\end{theorem}









\end{document}
