\documentclass[12pt,a4paper]{article}
\usepackage{times} 
\usepackage{amsmath}
\usepackage{amsfonts}
\textwidth160mm
\textheight247mm
\oddsidemargin0cm
\topmargin-0.5in
\pagestyle{empty}
\renewcommand{\baselinestretch}{1}
\begin{document}
\begin{center}
% TITLE
{\large{\bf Sofic entropy of some subshifts of finite type } }
% AUTHORS
\vskip0.5\baselineskip{\bf \underline {Bruno Seefeld}$^{1}$ }
% AFFILIATION
\vskip0.5\baselineskip{\em$^{1}$ Instituto Nacional de Matemática Pura e Aplicada}\\

\end{center}

\noindent

For an amenable group $G$, entropy theory was developed by Ornstein and 
Weiss in 
\cite{weiss}. Intuitevely, configurations on a Følner sequence (like $n$-balls around
the origin in $\mathbb{Z}^k$)
are counted and entropy exists by subbaditivity along it. When the group
is not amenable, like $\mathbb{F}_2$, sofic entropy theory, developed
by Bowen in \cite{bowen_1}, needs to be used. 
\\
Sofic entropy is a topological-dynamical
invariant of dynamical systems and in general is very difficult to
estimate. In this work the counting done by Ban et al. in \cite{ban}  of configurations
on trees is used to 
create microstates of an $\mathbb{F}_2$ action on a subshift of finite type on
$\mathcal{A}^{\mathbb{F}_2}$. The upper bound obtained on sofic entropy
is similar to the classical one of a $\mathbb{Z}$ SSFT.



\begingroup
\renewcommand{\section}[2]{}
\begin{thebibliography}{0}
\setlength{\parskip}{0mm}
\setlength{\itemsep}{-0.3mm}
\small
\bibitem{weiss} D. Ornstein, B. Weiss,  Entropy and isomorphism theorems for actions
of amenable groups  ,J. Anal. Math. \textbf{48}, 1–141 (1987).
\bibitem{bowen_1} L. Bowen, Measure conjugacy invariants for actions of countable sofic groups,
J. Amer. Math. Soc. \textbf{23}, 217-245. (2010).
\bibitem{ban} J. Ban, C. Chang, W. Hu, Y. Yu, Topological entropy for shifts of finite type over Z and trees,
Theoretical Computer Science. \textbf{930}, 24-32, (2022). 
\end{thebibliography}
\endgroup


\end{document}


