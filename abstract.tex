\documentclass[12pt,a4paper]{article}
\usepackage{times} 
\usepackage{amsmath}
\usepackage{amsfonts}
\textwidth160mm
\textheight247mm
\oddsidemargin0cm
\topmargin-0.5in
\pagestyle{empty}
\renewcommand{\baselinestretch}{1}
\begin{document}
\begin{center}
% TITLE
{\large{\bf Sofic entropy of some subshifts of finite type } }
% AUTHORS
\vskip0.5\baselineskip{\bf \underline {Bruno Seefeld}$^{1}$ }
% AFFILIATION
\vskip0.5\baselineskip{\em$^{1}$ Instituto Nacional de Matemática Pura e Aplicada}\\

\end{center}

\noindent

For an amenable group $G$, entropy theory was developed by (Weiss I
guess) in ref weiss. Intuitevely, configurations on a Følner sequence (like $n$-balls around
the origin in $\mathbb{Z}^k$)
are counted and entropy exists by subbaditivity along it. When the group
is not amenable, like $\mathbb{F}_2$, sofic entropy theory, developed
by Bowen, needs to be used. 
\\
Sofic entropy is a topological-dynamical
invariant of dynamical systems and in general is very difficult to
estimate. In this work the counting done by Ban e Chang of configurations
on trees is used to 
create microstates of an $\mathbb{F}_2$ action on a subshift of finite type on
$\mathcal{A}^{\mathbb{F}_2}$. This allows to recover a classical 
upper bound on entropy of a SSFT. 



\begingroup
\renewcommand{\section}[2]{}
\begin{thebibliography}{0}
\setlength{\parskip}{0mm}
\setlength{\itemsep}{-0.3mm}
\small
\bibitem{reference1} A. Author, B. Coauthor, J. Sci. Res. \textbf{13}, 1357 (2012).
\bibitem{reference2} A. Author, B. Coauthor, J. Sci. Res. \textbf{17}, 7531 (2013).
\end{thebibliography}
\endgroup


\end{document}


